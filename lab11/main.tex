\documentclass{article}

% Language setting
% Replace `english' with e.g. `spanish' to change the document language
\usepackage[utf8]{inputenc}
\usepackage[polish]{babel}
\usepackage[T1]{fontenc}

\usepackage{array}
\usepackage{booktabs}

% Set page size and margins
% Replace `letterpaper' with `a4paper' for UK/EU standard size
\usepackage[letterpaper,top=2cm,bottom=2cm,left=3cm,right=3cm,marginparwidth=1.75cm]{geometry}

% Useful packages
\usepackage{amsmath}
\usepackage{graphicx}
\usepackage[colorlinks=true, allcolors=blue]{hyperref}

\usepackage{float}


\usepackage{listings}
\lstset{
  inputencoding=utf8,
  extendedchars=true,
  literate=
    {ą}{{\k{a}}}1
    {ć}{{\'c}}1
    {ę}{{\k{e}}}1
    {ł}{{\l{}}}1
    {ń}{{\'n}}1
    {ó}{{\'o}}1
    {ś}{{\'s}}1
    {ź}{{\'z}}1
    {ż}{{\.z}}1
    {Ą}{{\k{A}}}1
    {Ć}{{\'C}}1
    {Ę}{{\k{E}}}1
    {Ł}{{\L{}}}1
    {Ń}{{\'N}}1
    {Ó}{{\'O}}1
    {Ś}{{\'S}}1
    {Ź}{{\'Z}}1
    {Ż}{{\.Z}}1,
}

\usepackage{xcolor}

\definecolor{codegreen}{rgb}{0,0.6,0}
\definecolor{codegray}{rgb}{0.5,0.5,0.5}
\definecolor{codepurple}{rgb}{0.58,0,0.82}
\definecolor{backcolour}{rgb}{0.95,0.95,0.92}

\lstdefinestyle{mystyle}{
    backgroundcolor=\color{backcolour},   
    commentstyle=\color{codegreen},
    keywordstyle=\color{magenta},
    numberstyle=\tiny\color{codegray},
    stringstyle=\color{codepurple},
    basicstyle=\ttfamily\footnotesize,
    breakatwhitespace=false,         
    breaklines=true,                 
    captionpos=b,                    
    keepspaces=true,                 
    numbers=left,                    
    numbersep=5pt,                  
    showspaces=false,                
    showstringspaces=false,
    showtabs=false,                  
    tabsize=2
}

\lstset{style=mystyle}

\usepackage{tikz}
\usetikzlibrary{trees}
\usepackage{forest}
\usetikzlibrary{arrows.meta, shapes.geometric, positioning}

\usepackage{authblk}


\title{Jednostka 11 - Paradygmat zdarzeniowy i reaktywny}
\author{Joanna Dagil}
\affil{Grupa TCH-1}

\begin{document}

\maketitle

\section{Cel ćwiczenia}

Celem ćwiczenia jest zapoznanie się z podstawami paradygmatu zdarzeniowego i reaktywnego. W szczególności oddzielenie źródła zdarzeń od ich obsługi (handlery, obserwatorzy, funkcje przetwarzające strumień). 


\section{Zadania}

\subsection*{Zadanie 1}

\lstinputlisting[language=Python]{1.py}

Przykładowy przebieg programu:

\begin{lstlisting}
Uwaga! Huragan o poziomie 24!
Uwaga! Pożar o poziomie 12!
Uwaga! Huragan o poziomie 18!
Uwaga! Pożar o poziomie 28!
Uwaga! Pożar o poziomie 32!
Uwaga! Burza o poziomie 59!
\end{lstlisting}

\subsection*{Zadanie 2}


\lstinputlisting[language=Python]{2.py}

Przykładowy przebieg programu:

\begin{lstlisting}
[LOG]
[UI] Aktualizacja danych:
     Temperatura: 27°C
     Wilgotność: 57%
     Ciśnienie: 1031 hPa
[LOG]
[Alarm] Temperatura przekroczyła 30°C
[Alarm] Wilgotność spadła poniżej 20%
[UI] Aktualizacja danych:
     Temperatura: 37°C
     Wilgotność: 7%
     Ciśnienie: 986 hPa
\end{lstlisting}



\subsection*{Zadanie 3}

\lstinputlisting[language=Python]{3.py}

Przykładowy przebieg programu:

\begin{lstlisting}
973
651
535
\end{lstlisting}

\section{Wnioski}


\subsection{Zalety podejścia zdarzeniowego}

\begin{itemize}
    \item \textbf{Modularność i luźne powiązanie} -- nowe handlery lub obserwatorów można łatwo dopinać lub odłączać bez zmiany kodu źródła zdarzeń. W zadaniu 1 wystarczy zarejestrować kolejną funkcję w \texttt{EventEmitter}, a w zadaniu 2 dodać nową klasę obserwatora, aby reagować na dodatkowe warunki pogodowe.
    \item \textbf{Naturalne modelowanie strumieni danych} -- w zadaniu 3 kolejne elementy strumienia są przetwarzane bez potrzeby trzymania całej kolekcji w pamięci.
    \item \textbf{Podział wywoływań i obsługi} -- rozdzielenie mechanizmów wywołania zdarzenia od jego obsługi.
\end{itemize}

\subsection{Ograniczenia podejścia zdarzeniowego}

\begin{itemize}
    \item \textbf{Trudniejsze śledzenie przepływu sterowania} -- kod reaguje na zdarzenia asynchronicznie, przez co trudniej prześledzić, co zostanie wykonane po kolei, co utrudnia testowanie i debugowanie.
    \item \textbf{Ryzyko braku obsługi zdarzeń} -- jeśli dla danego typu zdarzenia nie zarejestruje się handlera/obserwatora, zdarzenie jest po prostu ignorowane, co może prowadzić do trudnych do wykrycia błędów.
\end{itemize}


\subsection{Schematy i diagramy przebiegu zdarzeń w zadaniu 1 i 2}

\begin{figure}[h!]
    \centering
    \begin{minipage}{0.45\textwidth}
        \centering
        \includegraphics[width=\linewidth]{1.png}
        \caption{Schemat zdarzeń w zadaniu 1}
        \label{fig:zad1}
    \end{minipage}
    \hfill
    \begin{minipage}{0.50\textwidth}
        \centering
        \includegraphics[width=\linewidth]{2.png}
        \caption{Diagram klas obserwatora w zadaniu 2}
        \label{fig:zad2}
    \end{minipage}
\end{figure}



\end{document}
