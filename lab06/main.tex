\documentclass{article}

% Language setting
% Replace `english' with e.g. `spanish' to change the document language
\usepackage[utf8]{inputenc}
\usepackage[polish]{babel}
\usepackage[T1]{fontenc}

\usepackage{array}
\usepackage{booktabs}

% Set page size and margins
% Replace `letterpaper' with `a4paper' for UK/EU standard size
\usepackage[letterpaper,top=2cm,bottom=2cm,left=3cm,right=3cm,marginparwidth=1.75cm]{geometry}

% Useful packages
\usepackage{amsmath}
\usepackage{graphicx}
\usepackage[colorlinks=true, allcolors=blue]{hyperref}

\usepackage{float}


\usepackage{listings}
\lstset{
  inputencoding=utf8,
  extendedchars=true,
  literate=
    {ą}{{\k{a}}}1
    {ć}{{\'c}}1
    {ę}{{\k{e}}}1
    {ł}{{\l{}}}1
    {ń}{{\'n}}1
    {ó}{{\'o}}1
    {ś}{{\'s}}1
    {ź}{{\'z}}1
    {ż}{{\.z}}1
    {Ą}{{\k{A}}}1
    {Ć}{{\'C}}1
    {Ę}{{\k{E}}}1
    {Ł}{{\L{}}}1
    {Ń}{{\'N}}1
    {Ó}{{\'O}}1
    {Ś}{{\'S}}1
    {Ź}{{\'Z}}1
    {Ż}{{\.Z}}1,
}

\usepackage{xcolor}

\definecolor{codegreen}{rgb}{0,0.6,0}
\definecolor{codegray}{rgb}{0.5,0.5,0.5}
\definecolor{codepurple}{rgb}{0.58,0,0.82}
\definecolor{backcolour}{rgb}{0.95,0.95,0.92}

\lstdefinestyle{mystyle}{
    backgroundcolor=\color{backcolour},   
    commentstyle=\color{codegreen},
    keywordstyle=\color{magenta},
    numberstyle=\tiny\color{codegray},
    stringstyle=\color{codepurple},
    basicstyle=\ttfamily\footnotesize,
    breakatwhitespace=false,         
    breaklines=true,                 
    captionpos=b,                    
    keepspaces=true,                 
    numbers=left,                    
    numbersep=5pt,                  
    showspaces=false,                
    showstringspaces=false,
    showtabs=false,                  
    tabsize=2
}

\lstset{style=mystyle}

\usepackage{tikz}
\usetikzlibrary{trees}
\usepackage{forest}
\usetikzlibrary{arrows.meta, shapes.geometric, positioning}


\title{Your Paper}
\author{You}

\begin{document}

\textbf{Tytuł ćwiczenia:} Jednostka 6 - Paradygmat funkcyjny

\textbf{Autor:} Joanna Dagil

\textbf{Grupa:} TCH-1

\textbf{Data:} 18.11.2025

\section{Cel ćwiczenia}

Celem ćwiczenia jest zapoznanie się z podstawami paradygmatu funkcyjnego i porównanie go z programowaniem imperatywnym.


\section{Przebieg ćwiczenia}

W tym ćwiczeniu posłużę się językiem Python w edytorze Visual Studio Code. W celu wykonania zadań zapoznaję się z tutorialami, a następnie przechodzę do stworzenia pliku:

\vspace{2ex}

\textbf{lab06.py}
\lstinputlisting[language=Python]{przetworz.py}



Program uruchamiam poleceniem 

\begin{lstlisting}
python lab06.py
\end{lstlisting}




\section{Wyniki działania}

 Dla przykładowych danych

\begin{lstlisting}
Oryginalne liczby:     [1, 2, 3, 4, 5, 6, 7, 8, 9, 10]
-------------- ZADANIE 1 --------------
Przetworzone liczby:   [4, 16, 36, 64, 100]
-------------- ZADANIE 2 --------------
Agregacja sumowaniem:  55
Agregacja iloczynem:   3628800
Agregacja maksowaniem: 10
-------------- ZADANIE 3 --------------
Przed filtrowaniem:  10000
Po filtrowaniu:      4052
\end{lstlisting}


\section{Wnioski}

\subsection{Porównanie rozwiązań imperatywnych i funkcyjnych}


\begin{table}[h!]
\centering
\begin{tabular}{p{4cm} p{5cm} p{5cm}}
\toprule
\textbf{Kryterium} & \textbf{Imperatywne} & \textbf{Funkcyjne} \\
\midrule
\textbf{Paradygmat} & Skupia się na sekwencji instrukcji zmieniających stan programu & Skupia się na definiowaniu funkcji i przepływie danych między nimi \\
\textbf{Stan programu} & Zmienny – dane są modyfikowane w trakcie działania & Niezmienny – dane są niemutowalne, funkcje nie zmieniają stanu \\
\textbf{Podejście do obliczeń} & Wykonuje polecenia krok po kroku & Deklaruje, co ma być obliczone, bez opisywania kolejnych kroków \\
\textbf{Efekty uboczne} & Dozwolone i często wykorzystywane & Ograniczane lub eliminowane \\
\textbf{Powtarzanie} & Implementowane przy użyciu pętli & Implementowane przy użyciu rekurencji \\
\textbf{Zmienne, listy, obiekty} & Wykorzystywane & Jedyne jako parametry funkcji \\
\bottomrule
\end{tabular}
\end{table}

\subsection{Zalety i wady stylu funkcyjnego}

\textbf{Zalety:}
\begin{itemize}
    \item Kod jest bardziej zwięzły i deklaratywny – opisuje \emph{co} ma być wykonane, a nie \emph{jak}.
    \item Łatwo tworzyć czyste funkcje, które nie modyfikują danych wejściowych, co ułatwia testowanie.
    \item Zastosowanie \texttt{map}, \texttt{filter} i \texttt{reduce} pozwala pisać kod modularny
    \item Funkcje można łatwo przekazywać jako argumenty, co zwiększa elastyczność programu.
\end{itemize}

\noindent\textbf{Wady:}
\begin{itemize}
    \item Styl funkcyjny może być mniej czytelny dla osób przyzwyczajonych do programowania imperatywnego.
    \item Trudniejsze może być {śledzenie przepływu danych} i debugowanie złożonych funkcji anonimowych.
    \item Niektóre operacje (np. zagnieżdżone \texttt{lambda}) mogą być {mniej wydajne} lub trudniejsze do optymalizacji.
\end{itemize}

\noindent Czysto funkcyjne są wszystkie funkcje poza mainem.

\end{document}
