\documentclass{article}

% Language setting
% Replace `english' with e.g. `spanish' to change the document language
\usepackage[utf8]{inputenc}
\usepackage[polish]{babel}
\usepackage[T1]{fontenc}

% Set page size and margins
% Replace `letterpaper' with `a4paper' for UK/EU standard size
\usepackage[letterpaper,top=2cm,bottom=2cm,left=3cm,right=3cm,marginparwidth=1.75cm]{geometry}

% Useful packages
\usepackage{amsmath}
\usepackage{graphicx}
\usepackage[colorlinks=true, allcolors=blue]{hyperref}

\usepackage{float}


\usepackage{listings}
\lstset{
  inputencoding=utf8,
  extendedchars=true,
  literate=
    {ą}{{\k{a}}}1
    {ć}{{\'c}}1
    {ę}{{\k{e}}}1
    {ł}{{\l{}}}1
    {ń}{{\'n}}1
    {ó}{{\'o}}1
    {ś}{{\'s}}1
    {ź}{{\'z}}1
    {ż}{{\.z}}1
    {Ą}{{\k{A}}}1
    {Ć}{{\'C}}1
    {Ę}{{\k{E}}}1
    {Ł}{{\L{}}}1
    {Ń}{{\'N}}1
    {Ó}{{\'O}}1
    {Ś}{{\'S}}1
    {Ź}{{\'Z}}1
    {Ż}{{\.Z}}1,
}

\usepackage{xcolor}

\definecolor{codegreen}{rgb}{0,0.6,0}
\definecolor{codegray}{rgb}{0.5,0.5,0.5}
\definecolor{codepurple}{rgb}{0.58,0,0.82}
\definecolor{backcolour}{rgb}{0.95,0.95,0.92}

\lstdefinestyle{mystyle}{
    backgroundcolor=\color{backcolour},   
    commentstyle=\color{codegreen},
    keywordstyle=\color{magenta},
    numberstyle=\tiny\color{codegray},
    stringstyle=\color{codepurple},
    basicstyle=\ttfamily\footnotesize,
    breakatwhitespace=false,         
    breaklines=true,                 
    captionpos=b,                    
    keepspaces=true,                 
    numbers=left,                    
    numbersep=5pt,                  
    showspaces=false,                
    showstringspaces=false,
    showtabs=false,                  
    tabsize=2
}

\lstset{style=mystyle}

\usepackage{tikz}
\usetikzlibrary{trees}
\usepackage{forest}
\usetikzlibrary{arrows.meta, shapes.geometric, positioning}


\title{Your Paper}
\author{You}

\begin{document}

\textbf{Tytuł ćwiczenia:} Jednostka 5 - Paradygmat obiektowy (cz. 2)

\textbf{Autor:} Joanna Dagil

\textbf{Grupa:} TCH-1

\textbf{Data:} 04.11.2025

\section{Cel ćwiczenia}

Celem ćwiczenia jest zapoznanie się z dziedziczeniem, polimorfizmem i interfejsami w paradygmacie obiektowym.


\section{Przebieg ćwiczenia}

W tym ćwiczeniu posłużę się językiem Python w edytorze Visual Studio Code. W celu wykonania zadań zapoznaję się z tutorialami, a następnie przechodzę do stworzenia plików:

\vspace{2ex}

\textbf{pracownicy.py}
\lstinputlisting[language=Python]{pracownicy.py}
\indent \indent \textbf{figury.py}
\lstinputlisting[language=Python]{figury.py}
\textbf{pojazdy.py}
\lstinputlisting[language=Python]{pojazdy.py}


Program uruchamiam poleceniem 

\begin{lstlisting}
python {nazwa}.py
\end{lstlisting}




\section{Wyniki działania}

\indent \indent Dla \textbf{pracownicy.py}:

\begin{lstlisting}
10000
10000
15000
\end{lstlisting}

Dla \textbf{figury.py}:

\begin{lstlisting}
78.53981633974483
24
10.0
28.274333882308138
\end{lstlisting}

Dla \textbf{pojazdy.py}:

\begin{lstlisting}
Samochód: Toyota, drzwi: 5
Uruchamiam pojazd silnikowy...
Rower Merida, typ: górski
Uruchamiam pojazd...
Samochód: BMW, drzwi: 3
Uruchamiam pojazd silnikowy...
Motocykl: Yamaha, pojemność: 600cc
Uruchamiam pojazd silnikowy...
\end{lstlisting}

\section{Diagram hierarchii klas i opis polimorfizmu}

\begin{figure}[H]
    \centering
    \includegraphics[width=0.8\textwidth]{pracownicy.png}
    \caption{Diagram dla zadania 1}
\end{figure}

Polimorfizm w zadaniu 1. umożliwia traktowanie obiektów klas zarówno Programisty jak i Kierownika, jako obiektów typu Pracownik. Używamy tego do stworzenia listy pracownicy, w której mogą znajdować się zarówno Programiści, jak i Kierownicy. Możemy wtedy wywoływać tak samo nazwaną funkcję, ale otrzymywać inne wyniki dla każdego rodzaju obiektu.

\begin{figure}[H]
    \centering
    \includegraphics[width=0.8\textwidth]{figury.png}
    \caption{Diagram dla zadania 2}
\end{figure}

W tym przypadku polimorfizm ponownie pozwala nam na przechowywanie różnych rodzajów figur w jednej liście. Podobnie każda figura ma posiadać metodę zwracającą jej pole, natomiast mamy dodatkowo zabezpieczenie, że nie możliwe jest stworzenie niewspecyfikowanego rodzaju figury i żądanie jej pola - ma to sens z perspektywy geometrii.

\begin{figure}[H]
    \centering
    \includegraphics[width=0.8\textwidth]{pojazdy.png}
    \caption{Diagram dla zadania 3}
\end{figure}

Tutaj zastosowane jest dziedziczenie wielopoziomowe - pomiędzy klasą bazową, a klasami końcowymi istnieją także klasy pośrednie. Umożliwia to jeszcze optymalniejsze pogrupowanie klas i zmniejszenie powtarzalności kodu - gdyby klasa PojazdSilnikowy nie istniała, musielibyśmy nadpisać metodę uruchom() i w klasie Motocykl i w klasie Samochod. 


\section{Wnioski}

Programowanie obiektowe posługuje się polimorfizmem, a więc umożliwia nadpisywanie metod z klas, które są dziedziczone, przez klasy dziedziczące. Więc różne obiekty mogą reagować różnie, mimo, że pochodzą z tej samej klasy bazowej.

\end{document}
