\documentclass{article}

% Language setting
% Replace `english' with e.g. `spanish' to change the document language
\usepackage[utf8]{inputenc}
\usepackage[polish]{babel}
\usepackage[T1]{fontenc}

% Set page size and margins
% Replace `letterpaper' with `a4paper' for UK/EU standard size
\usepackage[letterpaper,top=2cm,bottom=2cm,left=3cm,right=3cm,marginparwidth=1.75cm]{geometry}

% Useful packages
\usepackage{amsmath}
\usepackage{graphicx}
\usepackage[colorlinks=true, allcolors=blue]{hyperref}

\usepackage{listings}
\lstset{
  inputencoding=utf8,
  extendedchars=true,
  literate=
    {ą}{{\k{a}}}1
    {ć}{{\'c}}1
    {ę}{{\k{e}}}1
    {ł}{{\l{}}}1
    {ń}{{\'n}}1
    {ó}{{\'o}}1
    {ś}{{\'s}}1
    {ź}{{\'z}}1
    {ż}{{\.z}}1
    {Ą}{{\k{A}}}1
    {Ć}{{\'C}}1
    {Ę}{{\k{E}}}1
    {Ł}{{\L{}}}1
    {Ń}{{\'N}}1
    {Ó}{{\'O}}1
    {Ś}{{\'S}}1
    {Ź}{{\'Z}}1
    {Ż}{{\.Z}}1,
}

\usepackage{xcolor}

\definecolor{codegreen}{rgb}{0,0.6,0}
\definecolor{codegray}{rgb}{0.5,0.5,0.5}
\definecolor{codepurple}{rgb}{0.58,0,0.82}
\definecolor{backcolour}{rgb}{0.95,0.95,0.92}

\lstdefinestyle{mystyle}{
    backgroundcolor=\color{backcolour},   
    commentstyle=\color{codegreen},
    keywordstyle=\color{magenta},
    numberstyle=\tiny\color{codegray},
    stringstyle=\color{codepurple},
    basicstyle=\ttfamily\footnotesize,
    breakatwhitespace=false,         
    breaklines=true,                 
    captionpos=b,                    
    keepspaces=true,                 
    numbers=left,                    
    numbersep=5pt,                  
    showspaces=false,                
    showstringspaces=false,
    showtabs=false,                  
    tabsize=2
}

\lstset{style=mystyle}

\usepackage{tikz}
\usetikzlibrary{trees}
\usepackage{forest}
\usetikzlibrary{arrows.meta, shapes.geometric, positioning}


\title{Your Paper}
\author{You}

\begin{document}

\textbf{Tytuł ćwiczenia:} Jednostka 4 - Paradygmat obiektowy (cz. 1)

\textbf{Autor:} Joanna Dagil

\textbf{Grupa:} TCH-1

\textbf{Data:} 28.10.2025

\section{Cel ćwiczenia}

Celem ćwiczenia jest zapoznanie się z zasadami paradygmatu obiektowego.


\section{Przebieg ćwiczenia}

W tym ćwiczeniu posłużę się językiem Python w edytorze Visual Studio Code. W celu wykonania zadań zapoznaję się z tutorialami, a następnie przechodzę do stworzenia plików:

\vspace{2ex}

\textbf{student.py}
\lstinputlisting[language=Python]{student.py}
\indent \indent \textbf{prostokat.py}
\lstinputlisting[language=Python]{prostokat.py}
\newpage
\textbf{bank.py}
\lstinputlisting[language=Python]{bank.py}


Program uruchamiam poleceniem 

\begin{lstlisting}
python {nazwa}.py
\end{lstlisting}




\section{Wyniki działania}

\indent \indent Dla \textbf{student.py}:

\begin{lstlisting}
Student Aaa Aaaaa ma średnia: 75.0
Student Bbb Bbbbb ma średnia: 25.0
Student Ccc Ccccc ma średnia: 65.0
Student Ccc Ccccc ma średnia: 69.0
\end{lstlisting}

Dla \textbf{prostokat.py}:

\begin{lstlisting}
Prostokąt 1 - pole: 12 obwód: 14
Prostokąt 2 - pole: 30 obwód: 22
Po skalowaniu Prostokąt 1 - pole: 48 obwód: 28
\end{lstlisting}

Dla \textbf{bank.py}:

\begin{lstlisting}
Saldo początkowe: 1000.0
Saldo po wpłacie 500.0: 1500.0
Saldo po wypłacie 200.0: 1300.0
Próba wypłaty 2000.0: Niepowodzenie
Saldo końcowe: 1300.0
\end{lstlisting}

\section{Opis struktury klas}

Wszystkie klasy w zadaniach dla tej jednostki nie dziedziczą, ani nie są dziedziczone, więc struktury są pojedyncze.

\section{Porównanie paradygmatu proceduralnego i obiektowego}

Paradygmat proceduralny oddziela dane od funkcji, które je przetwarzają, podczas gdy paradygmat obiektowy łączy je w jedną całość w postaci obiektów. W podejściu proceduralnym program składa się z sekwencji wywołań procedur, natomiast w obiektowym ze współpracujących obiektów. Proceduralny skupia się na operacjach wykonywanych na danych, a obiektowy na relacjach między bytami i ich zachowaniach. Funkcje przyporządkowane są do obiektów, co poprawia organizację kodu względem podejścia proceduralnego.

\section{Wnioski}

Paradygmat obiektowy opiera się na strukturach danych zwanymi obiektami. Umożliwia to naturalne podejście modelowania zagadnień. Pozwala nam raz definiować strukturę klasy i tworzyć (z użyciem konstruktora) dowolną ilość jej instancji, zapewniając ich kompatybilność.

\end{document}
