\documentclass{article}

% Language setting
% Replace `english' with e.g. `spanish' to change the document language
\usepackage[utf8]{inputenc}
\usepackage[polish]{babel}
\usepackage[T1]{fontenc}

\usepackage{array}
\usepackage{booktabs}

% Set page size and margins
% Replace `letterpaper' with `a4paper' for UK/EU standard size
\usepackage[letterpaper,top=2cm,bottom=2cm,left=3cm,right=3cm,marginparwidth=1.75cm]{geometry}

% Useful packages
\usepackage{amsmath}
\usepackage{graphicx}
\usepackage[colorlinks=true, allcolors=blue]{hyperref}

\usepackage{float}


\usepackage{listings}
\lstset{
  inputencoding=utf8,
  extendedchars=true,
  literate=
    {ą}{{\k{a}}}1
    {ć}{{\'c}}1
    {ę}{{\k{e}}}1
    {ł}{{\l{}}}1
    {ń}{{\'n}}1
    {ó}{{\'o}}1
    {ś}{{\'s}}1
    {ź}{{\'z}}1
    {ż}{{\.z}}1
    {Ą}{{\k{A}}}1
    {Ć}{{\'C}}1
    {Ę}{{\k{E}}}1
    {Ł}{{\L{}}}1
    {Ń}{{\'N}}1
    {Ó}{{\'O}}1
    {Ś}{{\'S}}1
    {Ź}{{\'Z}}1
    {Ż}{{\.Z}}1,
}

\usepackage{xcolor}

\definecolor{codegreen}{rgb}{0,0.6,0}
\definecolor{codegray}{rgb}{0.5,0.5,0.5}
\definecolor{codepurple}{rgb}{0.58,0,0.82}
\definecolor{backcolour}{rgb}{0.95,0.95,0.92}

\lstdefinestyle{mystyle}{
    backgroundcolor=\color{backcolour},   
    commentstyle=\color{codegreen},
    keywordstyle=\color{magenta},
    numberstyle=\tiny\color{codegray},
    stringstyle=\color{codepurple},
    basicstyle=\ttfamily\footnotesize,
    breakatwhitespace=false,         
    breaklines=true,                 
    captionpos=b,                    
    keepspaces=true,                 
    numbers=left,                    
    numbersep=5pt,                  
    showspaces=false,                
    showstringspaces=false,
    showtabs=false,                  
    tabsize=2
}

\lstset{style=mystyle}

\usepackage{tikz}
\usetikzlibrary{trees}
\usepackage{forest}
\usetikzlibrary{arrows.meta, shapes.geometric, positioning}

\usepackage{authblk}


\title{Jednostka 9 - Paradygmat logiczny}
\author{Joanna Dagil}
\affil{Grupa TCH-1}

\begin{document}

\maketitle

\section{Cel ćwiczenia}

Celem ćwiczenia jest zapoznanie się ze złożonymi zapytaniami, backtrackingiem i unifikacją w paradygmacie logicznym. Wykorzystujemy do tego język Prolog, w którym pokrywamy także zagadnienia takie jak rekurencja, struktury danych i operacje na nich.

\section{Zadania}

\subsection{Zadanie 1}

Tworzenie faktów \texttt{dziecko/2}

\begin{lstlisting}
dziecko(hela, anna).
dziecko(asia, anna).
dziecko(anna, olaf).
dziecko(paul, alek).
dziecko(fela, alek).
dziecko(alek, olaf).
dziecko(olaf, mela).
\end{lstlisting}

Diagram rodziny

\begin{figure}[h!]
    \centering
    \includegraphics[width=0.5\textwidth]{1.png}
\end{figure}

Tworzenie reguły \texttt{rodzeństwo/2}

\begin{lstlisting}
rodzenstwo(A, B) :- dziecko(A, Rodzic), dziecko(B, Rodzic).
\end{lstlisting}

Tworzenie reguły \texttt{dziadek/2}

Między dziadkiem a wnukiem musi być jedna instancja pośrednika - rodzica. 

\begin{lstlisting}
dziadek(Dziadek, Wnuk) :- dziecko(Rodzic, Dziadek), rodzic(Wnuk, Rodzic).
\end{lstlisting}

Tworzenie reguł \texttt{przodek/2}

Przodek może być albo bezpośrednio rodzicem potomka, albo rekurencyjnie rodzicem przodka potomka.

\begin{lstlisting}
przodek(Przodek, Potomek) :- dziecko(Potomek, Przodek).
przodek(Przodek, Potomek) :- dziecko(Posrednik, Przodek), przodek(Posrednik, Potomek).
\end{lstlisting}

\subsubsection{Przykłady zapytań prezentujące funkcjonowanie relacji rodzinnych}

Czy mela jest przodkiem feli?
\begin{lstlisting}
?- przodek(mela, fela).
true .
\end{lstlisting}
Jakich przodków ma fela?
\begin{lstlisting}
?- przodek(Kto, fela).
Kto = alek ;
Kto = olaf ;
Kto = mela .
\end{lstlisting}
Czyim przodkiem jest olaf?
\begin{lstlisting}
przodek(olaf, Kto).      
Kto = anna ;
Kto = alek ;
Kto = hela ;
Kto = asia ;
Kto = paul ;
Kto = fela .
\end{lstlisting}
Czy alek jest przodkiem olafa?
\begin{lstlisting}
?- przodek(alek, olaf).     
false.
\end{lstlisting}






\subsection{Zadanie 2}

Reguła \texttt{nalezy/2}

\begin{lstlisting}
nalezy(A, [A|_]).
nalezy(A, [_|T]) :- nalezy(A, T).
\end{lstlisting}

Regułą \texttt{dlugosc/2}

\begin{lstlisting}
dlugosc([], 0).
dlugosc([_|T], N) :- dlugosc(T, N2), N is N2 + 1.
\end{lstlisting}

\subsubsection{Przykłady zapytań prezentujące funkcjonowanie powyższych reguł}

\begin{lstlisting}
?- nalezy(3, [1,2,3,4]).
true.

?- nalezy(5, [1,2,3,4]).
false.

?- dlugosc([], N).
N = 0.

?- dlugosc([a,b,c,d], N).
N = 4.
\end{lstlisting}

\subsection{Zadanie 3}

Tworzenie grafu. \texttt{krawedz/2)}

\begin{lstlisting}
krawedz(1, 2).
krawedz(2, 3).
krawedz(3, 4).

krawedz(5, 6).
krawedz(6, 7).
studiuje(alek, matematyka).
\end{lstlisting}

Diagram grafu

\begin{figure}[H]
    \centering
    \includegraphics[width=0.5\textwidth]{2.png}
\end{figure}

Tworzenie reguły \texttt{sciezka/3}

\begin{lstlisting}
sciezka(Start, End, [Start, End]) :- krawedz(Start, End).
sciezka(Start, End, [Start|Tail]) :- krawedz(Start, X), sciezka(X, End, Tail).
\end{lstlisting}


\subsubsection{Przykłady zapytań prezentujące funkcjonowanie powyższych reguł}

\begin{lstlisting}
?- sciezka(1, 3, P).
P = [1, 2, 3].

?- sciezka(1, 2, P).
P = [1, 2].

?- sciezka(1, 5, P).
false.
\end{lstlisting}


\section{Wnioski}


Backtracking miał bezpośredni wpływ na sposób uzyskiwania wyników -- po zadaniu jednego zapytania Prolog automatycznie szukał kolejnych rozwiązań i po wciśnięciu \texttt{;} zwracał następne możliwe dopasowania. Było to wyraźnie widoczne przy wyszukiwaniu wszystkich przodków danej osoby oraz wszystkich ścieżek w grafie. Z jednej strony ułatwia to eksplorację wielu rozwiązań, z drugiej wymaga ostrożnego definiowania reguł (np. unikania cykli w grafie), aby obliczenia nie zapętlały się.


Unifikacja decydowała o tym, czy dane wywołanie predykatu pasuje do faktu lub reguły. Przy zapytaniach takich jak \texttt{przodek(Kto, fela)} czy \texttt{sciezka(1, 3, P)} Prolog automatycznie podstawiał wartości za zmienne tak, aby lewa i prawa strona dopasowania były zgodne. Dzięki temu nie trzeba ręcznie „przechodzić” po strukturach danych -- wystarczy zdefiniować ogólny wzorzec, a unifikacja zajmowała się doborem konkretnych wartości i wiązaniem zmiennych.
\end{document}
