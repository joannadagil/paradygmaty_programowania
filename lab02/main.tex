\documentclass{article}

% Language setting
% Replace `english' with e.g. `spanish' to change the document language
\usepackage[utf8]{inputenc}
\usepackage[polish]{babel}
\usepackage[T1]{fontenc}

% Set page size and margins
% Replace `letterpaper' with `a4paper' for UK/EU standard size
\usepackage[letterpaper,top=2cm,bottom=2cm,left=3cm,right=3cm,marginparwidth=1.75cm]{geometry}

% Useful packages
\usepackage{amsmath}
\usepackage{graphicx}
\usepackage[colorlinks=true, allcolors=blue]{hyperref}

\usepackage{listings}
\lstset{
  inputencoding=utf8,
  extendedchars=true,
  literate=
    {ą}{{\k{a}}}1
    {ć}{{\'c}}1
    {ę}{{\k{e}}}1
    {ł}{{\l{}}}1
    {ń}{{\'n}}1
    {ó}{{\'o}}1
    {ś}{{\'s}}1
    {ź}{{\'z}}1
    {ż}{{\.z}}1
    {Ą}{{\k{A}}}1
    {Ć}{{\'C}}1
    {Ę}{{\k{E}}}1
    {Ł}{{\L{}}}1
    {Ń}{{\'N}}1
    {Ó}{{\'O}}1
    {Ś}{{\'S}}1
    {Ź}{{\'Z}}1
    {Ż}{{\.Z}}1,
}

\usepackage{xcolor}

\definecolor{codegreen}{rgb}{0,0.6,0}
\definecolor{codegray}{rgb}{0.5,0.5,0.5}
\definecolor{codepurple}{rgb}{0.58,0,0.82}
\definecolor{backcolour}{rgb}{0.95,0.95,0.92}

\lstdefinestyle{mystyle}{
    backgroundcolor=\color{backcolour},   
    commentstyle=\color{codegreen},
    keywordstyle=\color{magenta},
    numberstyle=\tiny\color{codegray},
    stringstyle=\color{codepurple},
    basicstyle=\ttfamily\footnotesize,
    breakatwhitespace=false,         
    breaklines=true,                 
    captionpos=b,                    
    keepspaces=true,                 
    numbers=left,                    
    numbersep=5pt,                  
    showspaces=false,                
    showstringspaces=false,
    showtabs=false,                  
    tabsize=2
}

\lstset{style=mystyle}

\usepackage{tikz}
\usepackage{forest}
\usetikzlibrary{arrows.meta, shapes.geometric, positioning}


\title{Your Paper}
\author{You}

\begin{document}

\textbf{Tytuł ćwiczenia:} Jednostka 2 - Paradygmat proceduralny i strukturalny

\textbf{Autor:} Joanna Dagil

\textbf{Grupa:} TCH-1

\textbf{Data:} 14.10.2025

\section{Cel ćwiczenia}

Celem ćwiczenia jest zapoznanie się z zasadami paradygmatów proceduralnych i strukturalnych. A następnie refaktoryzacja programu imperatywnego z Jednostki 1 do wersji proceduralnej.


\section{Przebieg ćwiczenia}

W tym ćwiczeniu posłużę się językiem Python w edytorze Visual Studio Code. W celu wykonania zadań zapoznaję się z tutorialami, a następnie przechodzę do stworzenia plików:

\vspace{2ex}

\textbf{zad1.py} (dla zadania 1 i 2)

\lstinputlisting[language=Python]{zad1.py}




\vspace{3ex}

\textbf{zad3.py} (dla zadania 3)

\lstinputlisting[language=Python]{zad3.py}




Pliki uruchamiam poleceniem 

\begin{lstlisting}
python zad{N}.py
\end{lstlisting}




\section{Wyniki działania}

Dla Zadania 1 i 2:

\begin{lstlisting}
PS C:\Users\asiad\Desktop\paradygmaty\lab02> python zad1.py
Podaj liczbę elementów: 4
Podaj liczbę 1: 4.6
Podaj liczbę 2: 10.6
Podaj liczbę 3: 5
Podaj liczbę 4: 7.3
Posortowane dane: [4.6, 5.0, 7.3, 10.6]
Mediana vs średnia: 6.15 < 6.875
Średnia: 6.875, Min: 4.6, Max: 10.6
\end{lstlisting}


Dla Zadania 3:

\begin{lstlisting}
PS C:\Users\asiad\Desktop\paradygmaty\lab02> python zad3.py
Podaj liczbę ocen: 3
Podaj ocenę 1 (0-100): 40
Podaj ocenę 2 (0-100): 30
Podaj ocenę 3 (0-100): 10
Oceny powyżej średniej posortowane malejąco: 40 30 
Liczba ocen powyżej średniej: 2
PS C:\Users\asiad\Desktop\paradygmaty\lab02> python zad3.py
Podaj liczbę ocen: 4
Podaj ocenę 1 (0-100): 50
Podaj ocenę 2 (0-100): 40
Podaj ocenę 3 (0-100): 30
Podaj ocenę 4 (0-100): 40
Oceny powyżej średniej posortowane malejąco: 50 
Liczba ocen powyżej średniej: 1
PS C:\Users\asiad\Desktop\paradygmaty\lab02> python zad3.py
Podaj liczbę ocen: 0
Brak danych.
\end{lstlisting}

\newpage

\section{Schemat wywołań funkcji}

\begin{figure}[h!]
\centering

% Definicja stylów dla diagramu
\tikzstyle{startstop} = [rectangle, rounded corners, minimum width=3cm, minimum height=1cm,text centered, draw=black, fill=gray!20]
\tikzstyle{process} = [rectangle, minimum width=3cm, minimum height=1cm, text centered, draw=black, fill=blue!10]
\tikzstyle{arrow} = [thick,->,>=stealth]

\begin{tikzpicture}[node distance=1.7cm]

% Główna funkcja
\node (main) [startstop] {main()};

% Połączenia funkcji
\node (wczytaj) [process, below of=main, xshift=-8cm] {wczytaj\_dane()};
\node (oblicz) [process, below of=main, xshift=-4.8cm] {oblicz\_srednia()};
\node (min_max) [process, below of=main, xshift=-1.4cm] {znajdz\_min\_max()};
\node (znajdz) [process, below of=main, xshift=2cm] {znajdz\_mediane()};
\node (sort) [process, below of=main, xshift=5.2cm] {bubble\_sort()};
\node (drukuj) [process, right of=main, xshift=3.5cm] {drukuj\_wyniki()};


% Strzałki wywołań
\draw [arrow] (main) -- (wczytaj);
\draw [arrow] (main) -- (oblicz);
\draw [arrow] (main) -- (znajdz);
\draw [arrow] (main) -- (sort);
\draw [arrow] (main) -- (min_max);
\draw [arrow] (main) -- (drukuj);


% Dodatkowe elementy
\node (suma) [process, below of=oblicz] {oblicz\_sume()};
\draw [arrow] (oblicz) -- (suma);

\node (bub) [process, below of=znajdz] {bubble\_sort()};
\draw [arrow] (znajdz) -- (bub);



\end{tikzpicture}
\caption{Schemat blokowy wywołań funkcji w programie proceduralnym zad1.py.}
\end{figure}


\begin{figure}[h!]
\centering

% Definicja stylów dla diagramu
\tikzstyle{startstop} = [rectangle, rounded corners, minimum width=3cm, minimum height=1cm,text centered, draw=black, fill=gray!20]
\tikzstyle{process} = [rectangle, minimum width=3cm, minimum height=1cm, text centered, draw=black, fill=blue!10]
\tikzstyle{arrow} = [thick,->,>=stealth]

\begin{tikzpicture}[node distance=1.7cm]

% Główna funkcja
\node (main) [startstop] {main()};

% Połączenia funkcji
\node (wczytaj) [process, below of=main, xshift=-6cm] {wczytaj\_dane()};
\node (oblicz) [process, below of=main, xshift=-2.2cm] {oblicz\_srednia()};
\node (sort) [process, below of=main, xshift=1.8cm] {bubble\_sort\_malejący()};
\node (drukuj) [process, below of=main, xshift=6cm] {drukuj\_wyniki()};

% Strzałki wywołań
\draw [arrow] (main) -- (wczytaj);
\draw [arrow] (main) -- (oblicz);
\draw [arrow] (main) -- (sort);
\draw [arrow] (main) -- (drukuj);

% Dodatkowe elementy
\node (suma) [process, below of=oblicz] {oblicz\_sume()};
\draw [arrow] (oblicz) -- (suma);



\end{tikzpicture}
\caption{Schemat blokowy wywołań funkcji w programie strukturalnym zad3.py.}
\end{figure}

\section{Porównanie wersji imperatywnej i proceduralnej}

Wersja imperatywna programu opiera się na sekwencyjnym wykonywaniu instrukcji – program składa się głównie z ciągu poleceń modyfikujących zmienne i sterujących przepływem wykonania (np. poprzez pętle i instrukcje warunkowe). W takiej formie kod często ma charakter liniowy, co utrudnia jego czytelność, ponowne wykorzystanie fragmentów oraz utrzymanie w przypadku rozbudowy programu. Logika działania jest bezpośrednio „zaszyta” w głównym bloku programu.

Natomiast wersja strukturalna (proceduralna) wprowadza podział programu na mniejsze, logiczne jednostki – funkcje (procedury). Każda z nich realizuje określone zadanie, a główna funkcja \texttt{main()} koordynuje ich wywołania. Dzięki temu kod staje się bardziej przejrzysty, modularny i łatwiejszy w testowaniu oraz modyfikacji. W przypadku błędów lub potrzeby zmian wystarczy poprawić jedną funkcję, bez konieczności ingerencji w cały program.

Pod względem działania obie wersje realizują ten sam cel, jednak strukturalna forma pozwala na lepszą organizację kodu i w teorii eliminuje powtarzanie fragmentów.

\section{Wnioski}

Paradygmat proceduralny jest rozszerzeniem paradygmatu imperatywnego. Zwiększa czytelność i ułatwia podział zadań. Pozwala także na łatwiejszy przyszły rozwój oprogramowania - można edytować tylko poszczególną funkcję, bez ingerencji w cały kod. 

Jednocześnie nie jest to wziąć paradygmat wprowadzający niepotrzebnie złożoną strukturę, nawet w najprostszych programach i jest wygodny do małych i średnich projektów.

\end{document}
