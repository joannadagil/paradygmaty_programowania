\documentclass{article}

% Language setting
% Replace `english' with e.g. `spanish' to change the document language
\usepackage[utf8]{inputenc}
\usepackage[polish]{babel}
\usepackage[T1]{fontenc}

\usepackage{array}
\usepackage{booktabs}

% Set page size and margins
% Replace `letterpaper' with `a4paper' for UK/EU standard size
\usepackage[letterpaper,top=2cm,bottom=2cm,left=3cm,right=3cm,marginparwidth=1.75cm]{geometry}

% Useful packages
\usepackage{amsmath}
\usepackage{graphicx}
\usepackage[colorlinks=true, allcolors=blue]{hyperref}

\usepackage{float}


\usepackage{listings}
\lstset{
  inputencoding=utf8,
  extendedchars=true,
  literate=
    {ą}{{\k{a}}}1
    {ć}{{\'c}}1
    {ę}{{\k{e}}}1
    {ł}{{\l{}}}1
    {ń}{{\'n}}1
    {ó}{{\'o}}1
    {ś}{{\'s}}1
    {ź}{{\'z}}1
    {ż}{{\.z}}1
    {Ą}{{\k{A}}}1
    {Ć}{{\'C}}1
    {Ę}{{\k{E}}}1
    {Ł}{{\L{}}}1
    {Ń}{{\'N}}1
    {Ó}{{\'O}}1
    {Ś}{{\'S}}1
    {Ź}{{\'Z}}1
    {Ż}{{\.Z}}1,
}

\usepackage{xcolor}

\definecolor{codegreen}{rgb}{0,0.6,0}
\definecolor{codegray}{rgb}{0.5,0.5,0.5}
\definecolor{codepurple}{rgb}{0.58,0,0.82}
\definecolor{backcolour}{rgb}{0.95,0.95,0.92}

\lstdefinestyle{mystyle}{
    backgroundcolor=\color{backcolour},   
    commentstyle=\color{codegreen},
    keywordstyle=\color{magenta},
    numberstyle=\tiny\color{codegray},
    stringstyle=\color{codepurple},
    basicstyle=\ttfamily\footnotesize,
    breakatwhitespace=false,         
    breaklines=true,                 
    captionpos=b,                    
    keepspaces=true,                 
    numbers=left,                    
    numbersep=5pt,                  
    showspaces=false,                
    showstringspaces=false,
    showtabs=false,                  
    tabsize=2
}

\lstset{style=mystyle}

\usepackage{tikz}
\usetikzlibrary{trees}
\usepackage{forest}
\usetikzlibrary{arrows.meta, shapes.geometric, positioning}

\usepackage{authblk}


\title{Jednostka 8 - Paradygmat deklaratywny}
\author{Joanna Dagil}
\affil{Grupa TCH-1}

\begin{document}

\maketitle

\section{Cel ćwiczenia}

Celem ćwiczenia jest zapoznanie się z paradygmatem deklaratywnym i porównanie go z programowaniem imperatywnym i funkcyjnym. Wykorzystujemy do tego język SQL i Prolog.


\section{Zadania}

\subsection{Zadanie 1}

Tworzenie struktury tabeli.
\lstinputlisting[language=Python]{1structure.sql}

Wgrywanie przykładowych danych do tabeli.
\lstinputlisting[language=Python]{1data.sql}

Zapytania.
\lstinputlisting[language=Python]{1usage.sql}

Wyniki zapytań.

\begin{lstlisting}
+----+-----------+------------+------+-------------+
| id | imie      | nazwisko   | rok  | kierunek    |
+----+-----------+------------+------+-------------+
|  2 | Jan       | Nowak      |    3 | Informatyka |
|  4 | Krzysztof | Malinowski |    3 | Informatyka |
+----+-----------+------------+------+-------------+
2 rows in set (0.00 sec)

+-----------+-------------+--------------+
| imie      | nazwisko    | srednia_ocen |
+-----------+-------------+--------------+
| Anna      | Kowalska    |      4.75000 |
| Jan       | Nowak       |      3.75000 |
| Maria     | Wisniewska  |      4.00000 |
| Krzysztof | Malinowski  |      4.75000 |
+-----------+-------------+--------------+
4 rows in set (0.00 sec)

+-----------+------------+--------------+
| imie      | nazwisko   | srednia_ocen |
+-----------+------------+--------------+
| Anna      | Kowalska   |      4.75000 |
| Krzysztof | Malinowski |      4.75000 |
+-----------+------------+--------------+
2 rows in set (0.00 sec)
\end{lstlisting}


\subsection{Zadanie 2}

Tworzenie faktów \texttt{rodzic(dorosły, dziecko)}

\begin{lstlisting}
rodzic(jan, anna).
rodzic(jan, piotr).
rodzic(anna, kasia).
rodzic(anna, bartek).
rodzic(piotr, ola).
rodzic(piotr, tomek).
rodzic(maria, jan).
rodzic(maria, ewa).
\end{lstlisting}

Tworzenie reguły \texttt{dziecko/2}

\begin{lstlisting}
dziecko(Dziecko, Rodzic) :-
    rodzic(Rodzic, Dziecko).
\end{lstlisting}

Tworzenie reguł \texttt{przodek/2}

Przodek może być albo bezpośrednio rodzicem potomka, albo rekurencyjnie rodzicem przodka potomka.

\begin{lstlisting}
przodek(Przodek, Potomek) :-
    rodzic(Przodek, Potomek).

przodek(Przodek, Potomek) :-
    rodzic(Przodek, Posrednik),
    przodek(Posrednik, Potomek).
\end{lstlisting}

Przykłady zapytań prezentujące funkcjonowanie relacji rodzinnych

Kto jest rodzicem anny?
\begin{lstlisting}
?- rodzic(Kto, anna).
Kto = jan.
\end{lstlisting}
Kto jest dzieckiem anny?
\begin{lstlisting}
?- dziecko(Kto, anna).
Kto = kasia ;
Kto = bartek.
\end{lstlisting}
Czy jan jest przodkiem tomka?
\begin{lstlisting}
?- przodek(jan, tomek).
true.
\end{lstlisting}
Czy tomek jest przodkiem jana?
\begin{lstlisting}
?- przodek(tomek, jan).     
false.
\end{lstlisting}
Kto jest przodkiem tomka?
\begin{lstlisting}
?- przodek(Kto, tomek).
Kto = piotr ;
Kto = jan ;
Kto = maria.
\end{lstlisting}

\subsection{Zadanie 3}

Tworzenie faktów \texttt{studiuje(osoba, kierunek)}

\begin{lstlisting}
studiuje(hela, informatyka).
studiuje(asia, informatyka).
studiuje(fela, informatyka).
studiuje(olaf, informatyka).
studiuje(anna, matematyka).
studiuje(paul, matematyka).
studiuje(alek, matematyka).
\end{lstlisting}

Tworzenie reguły \texttt{studiowany/2}

\begin{lstlisting}
studiowany(Kierunek, Osoba) :-
    studiuje(Osoba, Kierunek).
\end{lstlisting}

Tworzenie faktów \texttt{prowadzi(osoba, kierunek)}

\begin{lstlisting}
prowadzi(adam, informatyka).
prowadzi(asia, matematyka).
\end{lstlisting}

Tworzenie reguły \texttt{prowadzony/2}

\begin{lstlisting}
prowadzony(Kierunek, Osoba) :-
    prowadzi(Osoba, Kierunek).
\end{lstlisting}

Tworzenie reguły \texttt{prowadzacy\_studenta/2}

\begin{lstlisting}
prowadzacy_studenta(Prowadzacy, Student) :-
    prowadzi(Prowadzacy, Kierunek),
    studiuje(Student, Kierunek).
\end{lstlisting}

Zapytanie "Kto jest prowadzącym studenta \texttt{alek}?"

\begin{lstlisting}
?- prowadzacy_studenta(Kto, alek).
Kto = asia.
\end{lstlisting}




\section{Wnioski}

Porównanie rozwiązań deklaratywnych i imperatywnych/funkcyjnych:


\begin{table}[h!]
\centering
\begin{tabular}{p{3cm} p{5.5cm} p{5.5
cm}}
\toprule
\textbf{Kryterium} & \textbf{Deklaratywne} & \textbf{Imperatywne/funkcyjne} \\
\midrule

\textbf{Sposób myślenia} 
& Myślimy w kategoriach faktów, reguł i własności, które mają być spełnione (\emph{co ma być prawdą?}) 
& Myślimy w kategoriach kroków algorytmu i przekształceń danych (\emph{jak to policzyć?}) \\

\textbf{Sposób opisu} 
& Fakty, reguły, zapytania; program to baza wiedzy i aparat wnioskowania 
& Funkcje, procedury, metody; program to zbiór instrukcji i definicji operujących na danych \\

\textbf{Przepływ} 
& Kolejność wykonania jest w dużej mierze ukryta – mechanizm wnioskowania steruje obliczeniami 
& Kolejność jest jawnie kontrolowana przez programistę (instrukcje, wywołania funkcji, pętle, rekursja) \\

\textbf{Wiele rozwiązań} 
& Naturalnie wspiera szukanie wszystkich wartości spełniających warunki jednego zapytania (wiele odpowiedzi na to samo pytanie) 
& Zwykle zwraca jedną wartość dla danego zestawu argumentów; wiele rozwiązań wymaga jawnego kodu (np. dodatkowych struktur danych) \\

\textbf{Powiązania między danymi} 
& Relacje między obiektami są centralne (predykaty typu \texttt{rodzic/2}, \texttt{student/3}); łatwo wyrażać złożone zależności i zapytania 
& Relacje trzeba zaszyć w strukturach danych i logice funkcji; zapytania są implementowane jako zwykły kod \\


\textbf{Zastosowania} 
& Systemy eksperckie, logika, wnioskowanie, zapytania do baz danych, zadania typu „znajdź wszystkie obiekty spełniające warunki” 
& Algorytmy numeryczne, przetwarzanie sygnałów, aplikacje interaktywne, systemy wbudowane, typowe programy użytkowe \\

\bottomrule
\end{tabular}
\end{table}




\end{document}
