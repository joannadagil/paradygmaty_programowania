\documentclass{article}

% Language setting
% Replace `english' with e.g. `spanish' to change the document language
\usepackage[utf8]{inputenc}
\usepackage[polish]{babel}
\usepackage[T1]{fontenc}

% Set page size and margins
% Replace `letterpaper' with `a4paper' for UK/EU standard size
\usepackage[letterpaper,top=2cm,bottom=2cm,left=3cm,right=3cm,marginparwidth=1.75cm]{geometry}

% Useful packages
\usepackage{amsmath}
\usepackage{graphicx}
\usepackage[colorlinks=true, allcolors=blue]{hyperref}

\usepackage{listings}
\lstset{
  inputencoding=utf8,
  extendedchars=true,
  literate=
    {ą}{{\k{a}}}1
    {ć}{{\'c}}1
    {ę}{{\k{e}}}1
    {ł}{{\l{}}}1
    {ń}{{\'n}}1
    {ó}{{\'o}}1
    {ś}{{\'s}}1
    {ź}{{\'z}}1
    {ż}{{\.z}}1
    {Ą}{{\k{A}}}1
    {Ć}{{\'C}}1
    {Ę}{{\k{E}}}1
    {Ł}{{\L{}}}1
    {Ń}{{\'N}}1
    {Ó}{{\'O}}1
    {Ś}{{\'S}}1
    {Ź}{{\'Z}}1
    {Ż}{{\.Z}}1,
}

\usepackage{xcolor}

\definecolor{codegreen}{rgb}{0,0.6,0}
\definecolor{codegray}{rgb}{0.5,0.5,0.5}
\definecolor{codepurple}{rgb}{0.58,0,0.82}
\definecolor{backcolour}{rgb}{0.95,0.95,0.92}

\lstdefinestyle{mystyle}{
    backgroundcolor=\color{backcolour},   
    commentstyle=\color{codegreen},
    keywordstyle=\color{magenta},
    numberstyle=\tiny\color{codegray},
    stringstyle=\color{codepurple},
    basicstyle=\ttfamily\footnotesize,
    breakatwhitespace=false,         
    breaklines=true,                 
    captionpos=b,                    
    keepspaces=true,                 
    numbers=left,                    
    numbersep=5pt,                  
    showspaces=false,                
    showstringspaces=false,
    showtabs=false,                  
    tabsize=2
}

\lstset{style=mystyle}


\title{Your Paper}
\author{You}

\begin{document}

\textbf{Tytuł ćwiczenia:} Jednostka 1 - Paradygmat imperatywny

\textbf{Autor:} Joanna Dagil

\textbf{Grupa:} TCH-1

\textbf{Data:} 07.10.2025

\section{Cel ćwiczenia}

Celem ćwiczenia jest zapoznanie się z klasyfikacją programowania na paradygmaty i z założeniami paradygmatu imperatywnego. A następnie zapoznanie się z kolejnymi elementami programowania imperatywnego:
\begin{enumerate}
\item zmiennymi i typami danych,
\item instrukcjami przypisania i sterującymi,
\item instrukcjami wejścia/wyjścia,
\item procedurami i funkcjami,
\item sekwencyjnością tego programowania.
\end{enumerate}


\section{Przebieg ćwiczenia}

W tym ćwiczeniu posłużę się językiem Python w edytorze Visual Studio Code. W celu wykonania zadań zapoznaję się z tutorialami, a następnie przechodzę do stworzenia plików:

\vspace{2ex}

\textbf{zad1.py}

\lstinputlisting[language=Python]{zad1.py}

\vspace{3ex}

\textbf{zad2.py}

\lstinputlisting[language=Python]{zad2.py}

\vspace{3ex}

\textbf{zad3.py}

\lstinputlisting[language=Python]{zad3.py}

Pliki uruchamiam poleceniem 

\begin{lstlisting}
python zad{N}.py
\end{lstlisting}


\section{Wyniki działania}

Dla Zadania 1:

\begin{lstlisting}
PS C:\Users\asiad\Desktop\paradygmaty> python zad1.py
Podaj liczbę całkowitą N: 3
Suma liczb parzystych od 0 do 3 wynosi 2
PS C:\Users\asiad\Desktop\paradygmaty> python zad1.py
Podaj liczbę całkowitą N: 5000
Suma liczb parzystych od 0 do 5000 wynosi 6252500
PS C:\Users\asiad\Desktop\paradygmaty> python zad1.py
Podaj liczbę całkowitą N: -20
Suma liczb parzystych od 0 do -20 wynosi 0
PS C:\Users\asiad\Desktop\paradygmaty> python zad1.py
Podaj liczbę całkowitą N: 0
Suma liczb parzystych od 0 do 0 wynosi 0
PS C:\Users\asiad\Desktop\paradygmaty> python zad1.py
Podaj liczbę całkowitą N: 1
Suma liczb parzystych od 0 do 1 wynosi 0
PS C:\Users\asiad\Desktop\paradygmaty> python zad1.py
Podaj liczbę całkowitą N: 1.5
Błąd: N musi być liczbą całkowitą!
\end{lstlisting}

Dla Zadania 2:

\begin{lstlisting}
PS C:\Users\asiad\Desktop\paradygmaty> python zad2.py
Podaj liczbę N: 16
16 15 14 13 12 11 10 9 8 7 6 5 4 3 2 1 
PS C:\Users\asiad\Desktop\paradygmaty> python zad2.py
Podaj liczbę N: -30

PS C:\Users\asiad\Desktop\paradygmaty> python zad2.py
Podaj liczbę N: 0
 
PS C:\Users\asiad\Desktop\paradygmaty> python zad2.py
Podaj liczbę N: 1
1 
\end{lstlisting}

Dla Zadania 3:

\begin{lstlisting}
PS C:\Users\asiad\Desktop\paradygmaty> python zad3.py
Podaj ilość liczb do sortowania: 3
Podaj 1 liczbę: 10
Podaj 2 liczbę: 3
Podaj 3 liczbę: 11
[3.0, 10.0, 11.0]
PS C:\Users\asiad\Desktop\paradygmaty> python zad3.py
Podaj ilość liczb do sortowania: 1
Podaj 1 liczbę: 2
[2.0]
PS C:\Users\asiad\Desktop\paradygmaty> python zad3.py
Podaj ilość liczb do sortowania: 4
Podaj 1 liczbę: 1.2
Podaj 2 liczbę: 1
Podaj 3 liczbę: 3.5
Podaj 4 liczbę: 1.2
[1.0, 1.2, 1.2, 3.5]
\end{lstlisting}


\section{Wnioski}

Paradygmat imperatywny wykonuje polecenia sekwencyjnie - po kolei, z wyjątkiem instrukcji sterujących - warunkowych lub pętli. Wartości zmiennych w trakcie trwania programu mogą ulegać zmianie - nazywane jest to "zmianą stanu programu".

W mojej ocenie jest to najbardziej podstawowy paradygmat, przeznaczony do małych, prostych projektów.

\end{document}
